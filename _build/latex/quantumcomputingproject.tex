%% Generated by Sphinx.
\def\sphinxdocclass{report}
\documentclass[letterpaper,10pt,english]{sphinxmanual}
\ifdefined\pdfpxdimen
   \let\sphinxpxdimen\pdfpxdimen\else\newdimen\sphinxpxdimen
\fi \sphinxpxdimen=.75bp\relax
\ifdefined\pdfimageresolution
    \pdfimageresolution= \numexpr \dimexpr1in\relax/\sphinxpxdimen\relax
\fi
%% let collapsible pdf bookmarks panel have high depth per default
\PassOptionsToPackage{bookmarksdepth=5}{hyperref}

\PassOptionsToPackage{booktabs}{sphinx}
\PassOptionsToPackage{colorrows}{sphinx}

\PassOptionsToPackage{warn}{textcomp}
\usepackage[utf8]{inputenc}
\ifdefined\DeclareUnicodeCharacter
% support both utf8 and utf8x syntaxes
  \ifdefined\DeclareUnicodeCharacterAsOptional
    \def\sphinxDUC#1{\DeclareUnicodeCharacter{"#1}}
  \else
    \let\sphinxDUC\DeclareUnicodeCharacter
  \fi
  \sphinxDUC{00A0}{\nobreakspace}
  \sphinxDUC{2500}{\sphinxunichar{2500}}
  \sphinxDUC{2502}{\sphinxunichar{2502}}
  \sphinxDUC{2514}{\sphinxunichar{2514}}
  \sphinxDUC{251C}{\sphinxunichar{251C}}
  \sphinxDUC{2572}{\textbackslash}
\fi
\usepackage{cmap}
\usepackage[T1]{fontenc}
\usepackage{amsmath,amssymb,amstext}
\usepackage{babel}



\usepackage{tgtermes}
\usepackage{tgheros}
\renewcommand{\ttdefault}{txtt}



\usepackage[Bjarne]{fncychap}
\usepackage{sphinx}

\fvset{fontsize=auto}
\usepackage{geometry}


% Include hyperref last.
\usepackage{hyperref}
% Fix anchor placement for figures with captions.
\usepackage{hypcap}% it must be loaded after hyperref.
% Set up styles of URL: it should be placed after hyperref.
\urlstyle{same}

\addto\captionsenglish{\renewcommand{\contentsname}{Contents:}}

\usepackage{sphinxmessages}
\setcounter{tocdepth}{1}



\title{Quantum Computing Project}
\date{Feb 23, 2024}
\release{0.01}
\author{
Conner Adlington, Julia Bauer, etc.\@{} 
}
\newcommand{\sphinxlogo}{\vbox{}}
\renewcommand{\releasename}{Release}
\makeindex
\begin{document}

\ifdefined\shorthandoff
  \ifnum\catcode`\=\string=\active\shorthandoff{=}\fi
  \ifnum\catcode`\"=\active\shorthandoff{"}\fi
\fi

\pagestyle{empty}
\sphinxmaketitle
\pagestyle{plain}
\sphinxtableofcontents
\pagestyle{normal}
\phantomsection\label{\detokenize{index::doc}}

\index{module@\spxentry{module}!example@\spxentry{example}}\index{example@\spxentry{example}!module@\spxentry{module}}

\chapter{Name The Chapter Up Here}
\label{\detokenize{index:name-the-chapter-up-here}}\label{\detokenize{index:module-example}}
\sphinxAtStartPar
This narrative can explain the whole point of the module (the file).

\sphinxAtStartPar
You can include maths inline like this: \(a^2 + b^2 = c^2\)

\sphinxAtStartPar
Or, if you need to display, do it like this (be sure to leave a blank line
above the .. math:: delcaration):
\begin{equation*}
\begin{split}\lim_{n\to\infty} \frac{1}{n} \neq \infty\end{split}
\end{equation*}
\sphinxAtStartPar
We can make use of referencing code objects like classes and methods
like this:

\sphinxAtStartPar
In this chapter we will start by explaining the {\hyperref[\detokenize{index:example.Example}]{\sphinxcrossref{\sphinxcode{\sphinxupquote{Example}}}}} class
and the required arguments for the \sphinxcode{\sphinxupquote{\_\_init\_\_()}} method.

\begin{sphinxadmonition}{note}{Note:}
\sphinxAtStartPar
Add notes like this.
\end{sphinxadmonition}

\begin{sphinxadmonition}{warning}{Warning:}
\sphinxAtStartPar
Add warnings like this.
\end{sphinxadmonition}

\sphinxAtStartPar
Insert code sample with backticks \sphinxtitleref{def \_\_init\_\_()}.
Make text \sphinxstylestrong{bold}, or \sphinxstyleemphasis{italics} just like markdown.
\begin{description}
\sphinxlineitem{This is a list:}\begin{itemize}
\item {} 
\sphinxAtStartPar
With

\item {} 
\sphinxAtStartPar
Bullets.

\end{itemize}

\sphinxlineitem{This is also a list:}\begin{enumerate}
\sphinxsetlistlabels{\arabic}{enumi}{enumii}{}{.}%
\item {} 
\sphinxAtStartPar
With

\item {} 
\sphinxAtStartPar
Numbers.

\end{enumerate}

\end{description}
\index{Example (class in example)@\spxentry{Example}\spxextra{class in example}}

\begin{fulllineitems}
\phantomsection\label{\detokenize{index:example.Example}}
\pysigstartsignatures
\pysiglinewithargsret{\sphinxbfcode{\sphinxupquote{class\DUrole{w}{ }}}\sphinxcode{\sphinxupquote{example.}}\sphinxbfcode{\sphinxupquote{Example}}}{\sphinxparam{\DUrole{n}{arg1}\DUrole{p}{:}\DUrole{w}{ }\DUrole{n}{int}}}{}
\pysigstopsignatures
\sphinxAtStartPar
Class explanation.

\sphinxAtStartPar
This can be a high level overview of what the object is.
\index{arg1 (example.Example attribute)@\spxentry{arg1}\spxextra{example.Example attribute}}

\begin{fulllineitems}
\phantomsection\label{\detokenize{index:example.Example.arg1}}
\pysigstartsignatures
\pysigline{\sphinxbfcode{\sphinxupquote{arg1}}}
\pysigstopsignatures
\sphinxAtStartPar
Describe the \sphinxtitleref{self.arg1} term.
This behaviour is unique to the init method.

\end{fulllineitems}

\index{usefulMethod() (example.Example method)@\spxentry{usefulMethod()}\spxextra{example.Example method}}

\begin{fulllineitems}
\phantomsection\label{\detokenize{index:example.Example.usefulMethod}}
\pysigstartsignatures
\pysiglinewithargsret{\sphinxbfcode{\sphinxupquote{usefulMethod}}}{\sphinxparam{\DUrole{n}{input}\DUrole{p}{:}\DUrole{w}{ }\DUrole{n}{int}}}{}
\pysigstopsignatures
\sphinxAtStartPar
Explain the utility of this method.
\begin{description}
\sphinxlineitem{Params:}\begin{description}
\sphinxlineitem{input:}
\sphinxAtStartPar
This \sphinxcode{\sphinxupquote{int}} should represent something.

\end{description}

\sphinxlineitem{Returns:}
\sphinxAtStartPar
This method returns a \sphinxcode{\sphinxupquote{Vector}}.

\end{description}

\end{fulllineitems}


\end{fulllineitems}



\bigskip\hrule\bigskip

\index{Example2 (class in example)@\spxentry{Example2}\spxextra{class in example}}

\begin{fulllineitems}
\phantomsection\label{\detokenize{index:example.Example2}}
\pysigstartsignatures
\pysigline{\sphinxbfcode{\sphinxupquote{class\DUrole{w}{ }}}\sphinxcode{\sphinxupquote{example.}}\sphinxbfcode{\sphinxupquote{Example2}}}
\pysigstopsignatures
\sphinxAtStartPar
This is the second class

\end{fulllineitems}

\index{module@\spxentry{module}!gates@\spxentry{gates}}\index{gates@\spxentry{gates}!module@\spxentry{module}}

\chapter{Gates Module}
\label{\detokenize{index:gates-module}}\label{\detokenize{index:module-gates}}
\begin{sphinxadmonition}{note}{Note:}
\sphinxAtStartPar
This code is still under development and until the version 
is incremented to a 1.* you should not trust any of this 
documentation.
\end{sphinxadmonition}
\index{Gate (class in gates)@\spxentry{Gate}\spxextra{class in gates}}

\begin{fulllineitems}
\phantomsection\label{\detokenize{index:gates.Gate}}
\pysigstartsignatures
\pysiglinewithargsret{\sphinxbfcode{\sphinxupquote{class\DUrole{w}{ }}}\sphinxcode{\sphinxupquote{gates.}}\sphinxbfcode{\sphinxupquote{Gate}}}{\sphinxparam{\DUrole{n}{matrix}\DUrole{p}{:}\DUrole{w}{ }\DUrole{n}{ndarray}}}{}
\pysigstopsignatures
\sphinxAtStartPar
This is the gate object which we will document properly now.

\begin{sphinxadmonition}{note}{Note:}
\sphinxAtStartPar
This class should not be called directly. It is a super
class for the gates below.
\end{sphinxadmonition}

\end{fulllineitems}

\index{H (class in gates)@\spxentry{H}\spxextra{class in gates}}

\begin{fulllineitems}
\phantomsection\label{\detokenize{index:gates.H}}
\pysigstartsignatures
\pysigline{\sphinxbfcode{\sphinxupquote{class\DUrole{w}{ }}}\sphinxcode{\sphinxupquote{gates.}}\sphinxbfcode{\sphinxupquote{H}}}
\pysigstopsignatures
\sphinxAtStartPar
Single\sphinxhyphen{}qubit Hadamard gate.

\sphinxAtStartPar
This gate is a \(\pi\) rotation about the X+Z axis, and has the effect of
changing computation basis from \(|0\rangle,|1\rangle\) to
\(|+\rangle,|-\rangle\) and vice\sphinxhyphen{}versa.

\sphinxAtStartPar
\sphinxstylestrong{Matrix Representation:}
\begin{equation*}
\begin{split}H = \frac{1}{\sqrt{2}}
    \begin{pmatrix}
        1 & 1 \\
        1 & -1
    \end{pmatrix}\end{split}
\end{equation*}
\end{fulllineitems}

\index{module@\spxentry{module}!tests.test\_tensor@\spxentry{tests.test\_tensor}}\index{tests.test\_tensor@\spxentry{tests.test\_tensor}!module@\spxentry{module}}

\chapter{Tensor Module Test Suite}
\label{\detokenize{index:tensor-module-test-suite}}\label{\detokenize{index:module-tests.test_tensor}}
\sphinxAtStartPar
This module tests the classes included in the tensor module.
\index{TestOperator (class in tests.test\_tensor)@\spxentry{TestOperator}\spxextra{class in tests.test\_tensor}}

\begin{fulllineitems}
\phantomsection\label{\detokenize{index:tests.test_tensor.TestOperator}}
\pysigstartsignatures
\pysiglinewithargsret{\sphinxbfcode{\sphinxupquote{class\DUrole{w}{ }}}\sphinxcode{\sphinxupquote{tests.test\_tensor.}}\sphinxbfcode{\sphinxupquote{TestOperator}}}{\sphinxparam{\DUrole{n}{methodName}\DUrole{o}{=}\DUrole{default_value}{\textquotesingle{}runTest\textquotesingle{}}}}{}
\pysigstopsignatures\index{test\_operator\_tensor\_product\_vs\_notes() (tests.test\_tensor.TestOperator method)@\spxentry{test\_operator\_tensor\_product\_vs\_notes()}\spxextra{tests.test\_tensor.TestOperator method}}

\begin{fulllineitems}
\phantomsection\label{\detokenize{index:tests.test_tensor.TestOperator.test_operator_tensor_product_vs_notes}}
\pysigstartsignatures
\pysiglinewithargsret{\sphinxbfcode{\sphinxupquote{test\_operator\_tensor\_product\_vs\_notes}}}{}{}
\pysigstopsignatures
\sphinxAtStartPar
Tensor Product of Hadamard, Identity and Hadamard, as presented in
the slides \sphinxhyphen{} \(H \otimes \mathbb{I} \otimes H\).

\sphinxAtStartPar
This manually creates the hadamard gates and performs the tensor
product using the \sphinxcode{\sphinxupquote{tensor()}} method.

\sphinxAtStartPar
The result is compared against the result in the slides:
\begin{equation*}
\begin{split}H \otimes \mathbb{I} \otimes H = \frac{1}{2}
    \begin{pmatrix}
        1 & 1 & 0 & 0 & 1 & 1 & 0 & 0 \\
        1 & -1 & 0 & 0 & 1 & -1 & 0 & 0 \\
        0 & 0 & 1 & 1 & 0 & 0 & 1 & 1 \\
        0 & 0 & 1 & -1 & 0 & 0 & 1 & -1 \\
        1 & 1 & 0 & 0 & -1 & -1 & 0 & 0 \\
        1 & -1 & 0 & 0 & 1 & -1 & 0 & 0 \\
        0 & 0 & 1 & 1 & 0 & 0 & -1 & -1 \\
        0 & 0 & 1 & -1 & 0 & 0 & -1 & 1
    \end{pmatrix}\end{split}
\end{equation*}
\end{fulllineitems}


\end{fulllineitems}

\index{TestVector (class in tests.test\_tensor)@\spxentry{TestVector}\spxextra{class in tests.test\_tensor}}

\begin{fulllineitems}
\phantomsection\label{\detokenize{index:tests.test_tensor.TestVector}}
\pysigstartsignatures
\pysiglinewithargsret{\sphinxbfcode{\sphinxupquote{class\DUrole{w}{ }}}\sphinxcode{\sphinxupquote{tests.test\_tensor.}}\sphinxbfcode{\sphinxupquote{TestVector}}}{\sphinxparam{\DUrole{n}{methodName}\DUrole{o}{=}\DUrole{default_value}{\textquotesingle{}runTest\textquotesingle{}}}}{}
\pysigstopsignatures
\sphinxAtStartPar
This class aims to provide end to end testing of the
Vector class within the tensor module.

\sphinxAtStartPar
The purpose of each test and the method by which it
is validated is outlined below.
\index{test\_vector\_construction() (tests.test\_tensor.TestVector method)@\spxentry{test\_vector\_construction()}\spxextra{tests.test\_tensor.TestVector method}}

\begin{fulllineitems}
\phantomsection\label{\detokenize{index:tests.test_tensor.TestVector.test_vector_construction}}
\pysigstartsignatures
\pysiglinewithargsret{\sphinxbfcode{\sphinxupquote{test\_vector\_construction}}}{}{}
\pysigstopsignatures
\sphinxAtStartPar
This test aims to check…

\end{fulllineitems}


\end{fulllineitems}



\renewcommand{\indexname}{Python Module Index}
\begin{sphinxtheindex}
\let\bigletter\sphinxstyleindexlettergroup
\bigletter{e}
\item\relax\sphinxstyleindexentry{example}\sphinxstyleindexpageref{index:\detokenize{module-example}}
\indexspace
\bigletter{g}
\item\relax\sphinxstyleindexentry{gates}\sphinxstyleindexpageref{index:\detokenize{module-gates}}
\indexspace
\bigletter{t}
\item\relax\sphinxstyleindexentry{tests.test\_tensor}\sphinxstyleindexpageref{index:\detokenize{module-tests.test_tensor}}
\end{sphinxtheindex}

\renewcommand{\indexname}{Index}
\printindex
\end{document}