%% Generated by Sphinx.
\def\sphinxdocclass{report}
\documentclass[letterpaper,10pt,english]{sphinxmanual}
\ifdefined\pdfpxdimen
   \let\sphinxpxdimen\pdfpxdimen\else\newdimen\sphinxpxdimen
\fi \sphinxpxdimen=.75bp\relax
\ifdefined\pdfimageresolution
    \pdfimageresolution= \numexpr \dimexpr1in\relax/\sphinxpxdimen\relax
\fi
%% let collapsible pdf bookmarks panel have high depth per default
\PassOptionsToPackage{bookmarksdepth=5}{hyperref}

\PassOptionsToPackage{booktabs}{sphinx}
\PassOptionsToPackage{colorrows}{sphinx}

\PassOptionsToPackage{warn}{textcomp}
\usepackage[utf8]{inputenc}
\ifdefined\DeclareUnicodeCharacter
% support both utf8 and utf8x syntaxes
  \ifdefined\DeclareUnicodeCharacterAsOptional
    \def\sphinxDUC#1{\DeclareUnicodeCharacter{"#1}}
  \else
    \let\sphinxDUC\DeclareUnicodeCharacter
  \fi
  \sphinxDUC{00A0}{\nobreakspace}
  \sphinxDUC{2500}{\sphinxunichar{2500}}
  \sphinxDUC{2502}{\sphinxunichar{2502}}
  \sphinxDUC{2514}{\sphinxunichar{2514}}
  \sphinxDUC{251C}{\sphinxunichar{251C}}
  \sphinxDUC{2572}{\textbackslash}
\fi
\usepackage{cmap}
\usepackage[T1]{fontenc}
\usepackage{amsmath,amssymb,amstext}
\usepackage{babel}



\usepackage{tgtermes}
\usepackage{tgheros}
\renewcommand{\ttdefault}{txtt}



\usepackage[Bjarne]{fncychap}
\usepackage{sphinx}

\fvset{fontsize=auto}
\usepackage{geometry}


% Include hyperref last.
\usepackage{hyperref}
% Fix anchor placement for figures with captions.
\usepackage{hypcap}% it must be loaded after hyperref.
% Set up styles of URL: it should be placed after hyperref.
\urlstyle{same}

\addto\captionsenglish{\renewcommand{\contentsname}{Contents:}}

\usepackage{sphinxmessages}
\setcounter{tocdepth}{1}



\title{QC Project}
\date{Feb 22, 2024}
\release{0.01}
\author{Group 5}
\newcommand{\sphinxlogo}{\vbox{}}
\renewcommand{\releasename}{Release}
\makeindex
\begin{document}

\ifdefined\shorthandoff
  \ifnum\catcode`\=\string=\active\shorthandoff{=}\fi
  \ifnum\catcode`\"=\active\shorthandoff{"}\fi
\fi

\pagestyle{empty}
\sphinxmaketitle
\pagestyle{plain}
\sphinxtableofcontents
\pagestyle{normal}
\phantomsection\label{\detokenize{index::doc}}

\index{module@\spxentry{module}!gates@\spxentry{gates}}\index{gates@\spxentry{gates}!module@\spxentry{module}}

\chapter{Gates Module}
\label{\detokenize{index:gates-module}}\label{\detokenize{index:module-gates}}
\begin{sphinxadmonition}{note}{Note:}
\sphinxAtStartPar
This code is still under development and until the version 
is incremented to a 1.* you should not trust any of this 
documentation.
\end{sphinxadmonition}
\index{Gate (class in gates)@\spxentry{Gate}\spxextra{class in gates}}

\begin{fulllineitems}
\phantomsection\label{\detokenize{index:gates.Gate}}
\pysigstartsignatures
\pysiglinewithargsret{\sphinxbfcode{\sphinxupquote{class\DUrole{w}{ }}}\sphinxcode{\sphinxupquote{gates.}}\sphinxbfcode{\sphinxupquote{Gate}}}{\sphinxparam{\DUrole{n}{matrix}\DUrole{p}{:}\DUrole{w}{ }\DUrole{n}{ndarray}}}{}
\pysigstopsignatures
\sphinxAtStartPar
This is the gate object which we will document properly now.

\begin{sphinxadmonition}{note}{Note:}
\sphinxAtStartPar
This class should not be called directly. It is a super
class for the gates below.
\end{sphinxadmonition}

\end{fulllineitems}

\index{H (class in gates)@\spxentry{H}\spxextra{class in gates}}

\begin{fulllineitems}
\phantomsection\label{\detokenize{index:gates.H}}
\pysigstartsignatures
\pysigline{\sphinxbfcode{\sphinxupquote{class\DUrole{w}{ }}}\sphinxcode{\sphinxupquote{gates.}}\sphinxbfcode{\sphinxupquote{H}}}
\pysigstopsignatures
\end{fulllineitems}

\index{module@\spxentry{module}!tests.test\_tensor@\spxentry{tests.test\_tensor}}\index{tests.test\_tensor@\spxentry{tests.test\_tensor}!module@\spxentry{module}}

\chapter{Tensor Module Test Suite}
\label{\detokenize{index:tensor-module-test-suite}}\label{\detokenize{index:module-tests.test_tensor}}
\sphinxAtStartPar
This module tests the classes included in the tensor module.
\index{TestOperator (class in tests.test\_tensor)@\spxentry{TestOperator}\spxextra{class in tests.test\_tensor}}

\begin{fulllineitems}
\phantomsection\label{\detokenize{index:tests.test_tensor.TestOperator}}
\pysigstartsignatures
\pysiglinewithargsret{\sphinxbfcode{\sphinxupquote{class\DUrole{w}{ }}}\sphinxcode{\sphinxupquote{tests.test\_tensor.}}\sphinxbfcode{\sphinxupquote{TestOperator}}}{\sphinxparam{\DUrole{n}{methodName}\DUrole{o}{=}\DUrole{default_value}{\textquotesingle{}runTest\textquotesingle{}}}}{}
\pysigstopsignatures
\end{fulllineitems}

\index{TestVector (class in tests.test\_tensor)@\spxentry{TestVector}\spxextra{class in tests.test\_tensor}}

\begin{fulllineitems}
\phantomsection\label{\detokenize{index:tests.test_tensor.TestVector}}
\pysigstartsignatures
\pysiglinewithargsret{\sphinxbfcode{\sphinxupquote{class\DUrole{w}{ }}}\sphinxcode{\sphinxupquote{tests.test\_tensor.}}\sphinxbfcode{\sphinxupquote{TestVector}}}{\sphinxparam{\DUrole{n}{methodName}\DUrole{o}{=}\DUrole{default_value}{\textquotesingle{}runTest\textquotesingle{}}}}{}
\pysigstopsignatures
\sphinxAtStartPar
This class aims to provide end to end testing of the
Vector class within the tensor module.

\sphinxAtStartPar
The purpose of each test and the method by which it
is validated is outlined below.
\index{test\_vector\_construction() (tests.test\_tensor.TestVector method)@\spxentry{test\_vector\_construction()}\spxextra{tests.test\_tensor.TestVector method}}

\begin{fulllineitems}
\phantomsection\label{\detokenize{index:tests.test_tensor.TestVector.test_vector_construction}}
\pysigstartsignatures
\pysiglinewithargsret{\sphinxbfcode{\sphinxupquote{test\_vector\_construction}}}{}{}
\pysigstopsignatures
\sphinxAtStartPar
This test aims to check…

\end{fulllineitems}


\end{fulllineitems}



\renewcommand{\indexname}{Python Module Index}
\begin{sphinxtheindex}
\let\bigletter\sphinxstyleindexlettergroup
\bigletter{g}
\item\relax\sphinxstyleindexentry{gates}\sphinxstyleindexpageref{index:\detokenize{module-gates}}
\indexspace
\bigletter{t}
\item\relax\sphinxstyleindexentry{tests.test\_tensor}\sphinxstyleindexpageref{index:\detokenize{module-tests.test_tensor}}
\end{sphinxtheindex}

\renewcommand{\indexname}{Index}
\printindex
\end{document}